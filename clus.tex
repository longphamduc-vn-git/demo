\documentclass[12pt,a4paper]{article}
\usepackage[utf8]{inputenc}
\usepackage[vietnamese]{babel}
\usepackage{amsmath,amssymb}
\usepackage{geometry}
\geometry{top=2cm,bottom=2cm,left=2cm,right=2cm}

\begin{document}

\textbf{Câu 1.} Đổi thứ tự lấy tích phân của tích phân kép
\[
I = \int_{-1}^{1} dx \int_{x^2}^{2-x} f(x,y)\,dy
\]

\begin{center}
\textbf{\underline{Giải}}
\end{center}

Miền tích phân:
\[
D: 
\begin{cases}
-1 \le x \le 1 \\
x^2 \le y \le 2-x
\end{cases}
\]

Xét giao điểm:
\[
x^2 = 2-x \Rightarrow x^2+x-2=0 \Rightarrow x=1,-2
\]
Trong đoạn $[-1,1]$ chỉ có $x=1$.

Ta chia miền theo biến $y$:

\[
D_1:
\begin{cases}
0 \le y \le 1 \\
-\sqrt{y} \le x \le \sqrt{y}
\end{cases}
\]

\[
D_2:
\begin{cases}
1 \le y \le 3 \\
-1 \le x \le 2-y
\end{cases}
\]

Vậy:
\[
I=
\int_{0}^{1} dy \int_{-\sqrt{y}}^{\sqrt{y}} f(x,y)\,dx
+
\int_{1}^{3} dy \int_{-1}^{2-y} f(x,y)\,dx
\]

\hrulefill

\textbf{Câu 2.} Dùng công thức Green tính tích phân đường
\[
I=\oint_{(C)} (x^3y-3y)\,dx+2xy^2\,dy
\]
trong đó $(C)$ là đường tròn $x^2+y^2=2y$ lấy theo chiều dương.

\begin{center}
\textbf{\underline{Giải}}
\end{center}

Ta có:
\[
P(x,y)=x^3y-3y,\quad Q(x,y)=2xy^2
\]

\[
\frac{\partial P}{\partial y}=x^3-3,\quad
\frac{\partial Q}{\partial x}=2y^2
\]

Theo công thức Green:
\[
I=\iint_D \left( \frac{\partial Q}{\partial x}
-\frac{\partial P}{\partial y}\right)\,dxdy
=\iint_D (2y^2-x^3+3)\,dxdy
\]

Phương trình đường tròn:
\[
x^2+y^2=2y
\Rightarrow x^2+(y-1)^2=1
\]
Vậy $D$ là hình tròn tâm $(0,1)$ bán kính $1$.

Ta tịnh tiến tọa độ:
\[
x=r\cos\theta,\quad
y=1+r\sin\theta
\]
\[
0\le r\le1,\quad 0\le\theta\le2\pi
\]

Tách tích phân:
\[
I=\iint_D 2y^2\,dxdy
-\iint_D x^3\,dxdy
+\iint_D 3\,dxdy
\]

Do miền đối xứng qua trục $x=0$:
\[
\iint_D x^3\,dxdy=0
\]

Diện tích hình tròn:
\[
\iint_D 3\,dxdy=3\pi
\]

Tính
\[
\iint_D 2y^2\,dxdy
=
2\iint_D (1+r\sin\theta)^2 r\,drd\theta
\]

Sau khi tính toán thu được:
\[
\iint_D 2y^2\,dxdy=3\pi
\]

Suy ra:
\[
I=3\pi+3\pi=6\pi
\]

\[
\boxed{I=6\pi}
\]

\hrulefill

\textbf{Câu 3.} Cho tích phân đường loại 2
\[
I=\int_{AB}(3x^2+a^2xy+1)\,dx+(3ax^2+y+1)\,dy
\]

\begin{itemize}
\item Tìm $a$ để tích phân không phụ thuộc đường đi.
\item Tính $I$ với $A(0,-1)$, $B(1,1)$ và $a=6$.
\end{itemize}

\begin{center}
\textbf{\underline{Giải}}
\end{center}

\[
P=3x^2+a^2xy+1,\quad
Q=3ax^2+y+1
\]

\[
P_y=a^2x,\quad Q_x=6ax
\]

Điều kiện độc lập đường đi:
\[
a^2x=6ax \Rightarrow a^2=6a
\Rightarrow a=0 \text{ hoặc } a=6
\]

Với $a=6$, tích phân không phụ thuộc đường đi.

Chọn đường gãy khúc:
\[
A(0,-1)\to C(0,1)\to B(1,1)
\]

Đoạn $AC$: $x=0$, $dx=0$
\[
\int_{AC}=\int_{-1}^{1}(y+1)\,dy=2
\]

Đoạn $CB$: $y=1$, $dy=0$
\[
\int_{CB}=\int_{0}^{1}(3x^2+36x+1)\,dx=20
\]

Vậy:
\[
\boxed{I=22}
\]

\end{document}
